%\documentclass{amsproc}
%\documentclass[reqno,11pt]{amsart}
%\documentclass[10pt]{amsart}
\documentclass[twoside, 12pt,a4paper]{article}
\usepackage{epsfig,amscd,amssymb,amsmath,amsfonts,csquotes,comment}
\usepackage{tcolorbox,fancyhdr}
\pagestyle{fancy}
\fancyhf{}

\newcommand{\Class}{CLASS TITLE}
\newcommand{\Lesson}{LESSON TITLE}


% Length to control the \fancyheadoffset and the calculation of \headline
% simultaneously
\newlength\FHoffset
\setlength\FHoffset{1cm}

\addtolength\headwidth{2\FHoffset}

\fancyheadoffset{\FHoffset}

% these lengths will control the headrule trimming to the left and right 
\newlength\FHleft
\newlength\FHright

% here the trimmings are controlled by the user
\setlength\FHleft{1cm}
\setlength\FHright{0cm}

% The new definition of headrule that will take into acount the trimming(s)
\newbox\FHline
\setbox\FHline=\hbox{\hsize=\paperwidth%
  \hspace*{\FHleft}%
  \rule{\dimexpr\headwidth-\FHleft-\FHright\relax}{\headrulewidth}\hspace*{\FHright}%
}
\renewcommand\headrule{\vskip-.7\baselineskip\copy\FHline}
\fancyhead[CE,CO]{\Lesson}
\fancyfoot[RE,LO]{Page \thepage}

\renewcommand{\abstractname}{Main Idea}

\usepackage{amsmath,titlesec}
\usepackage{graphicx}
\usepackage{lscape}
\usepackage{amsthm,color,cancel}
\renewcommand{\arraystretch}{1.5}

% Customizing the \section command
\titleformat{\section}
  {\normalfont\Large\bfseries} % Format for the section title
  {Activity \#\thesection:}       % Prefix before the section number
  {1em}                        % Space between the prefix and the title
  {}                           % Code preceding the title

% Customizing the \subsection command
\titleformat{\subsection}
  {\normalfont\bfseries} % Format for the section title
  {Model \thesection:}       % Prefix before the section number
  {1em}                        % Space between the prefix and the title
  {}                           % Code preceding the title

\usepackage[margin=1.05in]{geometry}
\usepackage[colorlinks]{hyperref}
\usepackage{siunitx}

\title{Topic: \Lesson}
\author{}
\date{}

\begin{document}
\maketitle
\thispagestyle{fancy} % so we get everything on page 1

\begin{center}

    \begin{tabular}{|>{\centering\arraybackslash}p{7cm}|>{\centering\arraybackslash}p{7cm}|}
       \hline
       \Large \textbf{Role}&\Large \textbf{Team Member}\\
       \hline 
       \textit{Manager}: assigns the roles \& makes sure everyone contributes appropriately.  & \\
       \hline 
       \textit{Presenter}: talks to the instructor and other teams. &\\
       \hline
       \textit{Recorder}: records all answers \& questions, and provides team reflection to team \& instructor. &\\
       \hline
       \textit{Reader}: keeps everyone on track \& reads the questions. &\\
       \hline
    \end{tabular}
\end{center}

\section*{\textbf{Learning Objectives:}}
\begin{enumerate}
    \item 
\end{enumerate}
\newpage
\section{Model Title}
\begin{tcolorbox}[colback=black!10, colframe=yellow, title=]
\subsection{}
This is the text inside the shaded box. You can add more text here to fit your needs.
\[f(x)=\begin{cases}
    x\\
    x^2
\end{cases}\]
\end{tcolorbox}


\end{document}
